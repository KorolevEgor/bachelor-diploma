В странах с большой железнодорожной сетью и большим потоком перемещения поездов, таких как РФ, США, Китай, Индия существует проблема схода составов с рельс, которые могут быть обусловленны различными факторами, их можно классифицировать на:

\begin{description}[font=$\bullet$]
\item внешние: кривизна пути, профиль пути, состояние транспортного пути, проблемы со стрелочным переводом, погодные условия (при экстремальных температурах рельсы могут сильно расширяться или сжиматься);
    
\item внутренние: количество вагонов в составе, загруженность, скорость, невнимательность машиниста, состояние состава.
\end{description}
Некоторые пути могут проходить через национальные парки, национальные заповедники и другие типы особо охраняемых объектов. По этой причине аварии, произошедшие на таких участках могут привести к экологической катастрофе, особенно велика опасность, если поезд был грузовым и перевозил легко воспламеняемые объекты (нефть, газ, метан, уголь, древесина) или высокотоксичные грузы. Следует отметить, что помимо экологической проблемы могут возникнуть и другие проблемы, например, такие как:

\begin{description}[font=$\bullet$]
\item логистическая - если состав сошел с рельс, следующим поездам приходится идти в обход, в некоторых случаях обхода может не быть;
\item экономическая - связанна с издержками транспортной компании по решению экологической проблемы, потери части вагонов, локомотива, утрата части груза, временные издержки;
\item инфраструктурная - повреждение строения железнодорожного пути, стыков, моста, обрушение тоннеля и др.
\end{description}
В данной работе рассматривается проблема схода состава с рельс, поскольку данная проблема является одной из самых опасных. В зависимости от масштаба происшествия сходы классифицируют на аварии и крушения. Согласно \cite{Ignatov:functional_dependence} за период с 2013 г. по 2016 г. в Российской Федерации имеется 262 протокола сходов с рельс вагонов как в грузовых поездах, так и в пассажирских поездах, без учета протоколов транспортных происшествий, классифицированных как крушения. Соответственно, при вычислении среднего числа дней без аварий выходит 4 дня, поэтому проблема представляет интерес для железнодорожных компаний.

В данной работе будет проведен анализ причин схода железнодорожного подвижного состава, а также будут построены предсказательные модели числа сошедших вагонов. Для достижения поставленных задач будут использованы методы теории вероятностей и математической статистики.

