Выпускная квалификационная работа содержит \pageref{LastPage}~страниц, \totfig~рисунков, \tottab~таблиц, список использованных источников содержит 15 позиций, 2 приложения.



\noindent
МЕТОД МАКСИМАЛЬНОГО ПРАВДОПОДОБИЯ, ПУАССОНОВСКАЯ РЕГРЕССИЯ, ГЕОМЕТРИЧЕСКАЯ РЕГРЕССИЯ, ГЛОБАЛЬНАЯ ОПТИМИЗАЦИЯ, ИНФОРМАЦИОННЫЕ КРИТЕРИИ




В выпускной квалификационной работе показано решение задачи регрессии. Имеется набор данных о случаях сходов с рельсов и крушений грузовых поездов по причине излома боковой рамы. Признак количество подвижных единиц в сходе является целевым. Ключевой особенностью набора данных является малая мощность выборки, а также его разреженность.

Для построения предсказательных моделей использовался метод максимального правдоподобия. Делается предположение о распределении целевого признака (геометрическое, либо Пуассоновское распределение). Целевой признак имеет слабую корреляцию с остальными признаками, по этой причине были введены новые признаки на основе имеющихся.

В работе были рассмотренны модели с различными признаковыми пространствами (8 пространств) и различными функциями связи (7 для Пуассоновской регрессии и 5 для геометрической). Было построено 56 моделей Пуассоновской регрессии и 40 моделей геометрической регрессии.

Скорректированный критерий Акаике $AIC_c$ -- показатель качества. Проведя численные эксперименты, были получены следующие результаты:
\begin{description}[font=$\bullet$]
    \item диапазон значений $AIC_c$ для Пуассоновской регрессии:  $[356.87, 567.74]$;
    \item диапазон значений $AIC_c$ для геометрической регрессии: $[153.65, 918.43]$.
\end{description}
Многие модели геометрической регрессии для заданного набора данных оказалась существенно лучше моделей пуассоновской регрессии.


В приложении к работе был разработан программный комплекс (веб-сервис), реализующий работу с методом максимального правдоподобия. Использованы следующие языки и технологии: Python, Java, Spring (Boot, AOP, Web, WebSocket, Data, Security), Log4J, JUnit, Mockito, MyBatis, FlyWay, PostgreSQL, Angular, HTML, CSS, Bootstrap, TypeScript.




