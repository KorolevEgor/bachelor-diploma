\begin{thebibliography}{00}
    
    \bibitem{Ignatov:functional_dependence} Замышляев А.М., Игнатов А.Н., Кибзун А.И., Новожилов Е.О. Функциональная зависимость между количеством вагонов в сходе из-за неисправностей вагонов или пути и факторами движения // Надежность. 2018. Т. 18, № 1. С….. DOI: 10.21683/1729-2646-2018-18-1…
    
    \bibitem{coursera:andrew_ng} Andrew Ng., Machine Learning from Stanford University. https://www.coursera.org/learn/machine-learning
    
    \bibitem{coursera:voroncov} Воронцов К.В., Введение в машинное обучение от НИУ ВШЭ \& Yandex School of Data Analysis. https://www.coursera.org/learn/vvedenie-mashinnoe-obuchenie
    
    \bibitem{wiki:poisson_regression} Пуассоновская регрессия. https://en.wikipedia.org/wiki/Poisson\_regression
    
    \bibitem{towardsdatascience:poisson_regression} Пуассоновская регрессия. https://towardsdatascience.com/an-illustrated-guide-to-the-poisson-regression-model-50cccba15958
    
    \bibitem{Samuel:ML} Samuel, Arthur L. Some Studies in Machine Learning Using the Game of Checkers // IBM Journal. - 1959. - №3. - http://www.cs.virginia.edu/~evans/greatworks/samuel1959.pdf
    
    \bibitem{wiki:crisp_dm} CRISP-DM. https://en.wikipedia.org/wiki/Cross-industry\_standard\_process\_for\_data\_mining
    
    \bibitem{poll:crisp_dm} Методологии анализа данных. https://www.kdnuggets.com/2014/10/crisp-dm-top-methodology-analytics-data-mining-data-science-projects.html
    
    \bibitem{stepik:matstat_1} Основы статистики часть 1. https://stepik.org/course/76/info
    
    \bibitem{stepik:matstat_2} Основы статистики часть 2. https://stepik.org/course/524/info
    
    \bibitem{mlwiki:normal_equ} Normal Equation. http://mlwiki.org/index.php/Normal\_Equation
    
    \bibitem{habr:grad_descent} Обзор градиентных методов в задачах математической оптимизации. https://habr.com/ru/post/413853/
    
    \bibitem{wiki:stochastic_grad} Метод стохастической аппроксимации. https://en.wikipedia.org/wiki/Stochastic\_approximation
    
    \bibitem{ml:regression_calc} Регрессионный анализ. http://www.machinelearning.ru/wiki/ index.php?title=Регрессионный\_анализ
    
    \bibitem{wiki:overfitting} Эффект переобучения. https://en.wikipedia.org/wiki/Overfitting
    
    \bibitem{habr:cart} Деревья решений для решения задач регрессии. https://habr.com/ru/post/116385/ 
    
\end{thebibliography}
