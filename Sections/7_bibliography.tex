\begin{thebibliography}{00}
    
    \bibitem{Ignatov:functional_dependence} Замышляев А.М., Игнатов А.Н., Кибзун А.И., Новожилов Е.О. Функциональная зависимость между количеством вагонов в сходе из-за неисправностей вагонов или пути и факторами движения // Надежность. -- 2018. -- С.1-15.
    
    \bibitem{Samuel:ML} Samuel Arthur. Some Studies in Machine Learning Using the Game of Checkers // IBM Journal. -- 1959. -- №3. -- 20 с. [Electronic resource]. URL: http://www.cs.virginia.edu/~evans/greatworks/samuel1959.pdf (date of treatment: 25.11.2021).
    
    \bibitem{coursera:andrew_ng} Andrew Ng. Machine Learning // Stanford Online [Electronic resource]. URL: https://www.coursera.org/learn/machine-learning (date of treatment: 03.12.2021).
    
    \bibitem{wiki:poisson_regression} Xiao Zhang. Poisson Regression // Microsoft Documentation. -- 2019. [Electronic resource]. URL: https://docs.microsoft.com/en-us/previous-versions/azure/machine-learning/studio-module-reference/poisson-regression (date of treatment: 02.12.2021).
    
    \bibitem{towardsdatascience:poisson_regression} Sachin Date. An Illustrated Guide to the Poisson Regression Model // Towards Data Science. -- 2019. [Electronic resource]. URL: https://towardsdatascience.com/an-illustrated-guide-to-the-poisson-regression-model-50cccba15958 (date of treatment: 10.12.2021).
    
    \bibitem{poll:crisp_dm} Gregory Piatetsky. CRISP-DM, still the top methodology for analytics, data mining, or data science projects // KDnuggets. -- 2014. [Electronic resource]. URL: https://www.kdnuggets.com/2014/10/crisp-dm-top-methodology-analytics-data-mining-data-science-projects.html (date of treatment: 15.12.2021).
    
    \bibitem{mlwiki:normal_equ} Alexey Grigorev. Normal Equation [Electronic resource]. URL: http://mlwiki.org/index.php/Normal\_Equation (date of treatment: 15.12.2021).
    
    \bibitem{coursera:voroncov} Воронцов К.В. Введение в машинное обучение // НИУ ВШЭ, Yandex School of Data Analysis [Электронный ресурс]. URL: https://www.coursera.org/learn/vvedenie-mashinnoe-obuchenie (дата обращения: 04.12.2021).
    
    \bibitem{wiki:crisp_dm} Константин Коточигов. CRISP-DM. // ГК ЛАНИТ. -- 2019. [Электронный ресурс] URL: https://habr.com/ru/company/lanit/blog/328858/ (дата обращения: 16.12.2021).
    
    \bibitem{stepik:matstat_1} Анатолий Карпов. Основы статистики. // Институт биоинформатики. -- 2020. [Электронный ресурс]. URL: https://stepik.org/course/76/info (дата обращения: 04.12.2021).
        
    \bibitem{habr:grad_descent} Николай Мальковский. Обзор градиентных методов в задачах математической оптимизации // 2018. [Электронный ресурс]. URL: https://habr.com/ru/post/413853/ (дата обращения: 16.12.2021).
    
    \bibitem{wiki:stochastic_grad} Юрий Кашницкий. Метод стохастической аппроксимации // Open Data Science. -- 2017. [Электронный ресурс]. URL: https://habr.com/ru/company/ods/blog/326418/ (дата обращения: 21.12.2021).
    
    \bibitem{ml:regression_calc} Регрессионный анализ [Электронный ресурс]. URL: http://www.machinelearning.ru/wiki/index.php?title=Регрессионный\_анализ // 2016. (дата обращения: 03.12.2021).
    
    \bibitem{wiki:overfitting} Воронцов К.В. Недообучение и переобучение в машинном интеллекте // ПостНаука. -- 2020. [Электронный ресурс]. URL: https://postnauka.ru/video/154955 (дата обращения: 03.12.2021).
    
    \bibitem{habr:cart} Илья Полосухин. Классификация и регрессия с помощью деревьев принятия решений // 2011. [Электронный ресурс]. URL: https://habr.com/ru/post/116385/ (дата обращения: 03.12.2021).
    
\end{thebibliography}
