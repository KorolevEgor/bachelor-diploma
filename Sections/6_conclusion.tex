В данной работе была рассмотрена проблема схода подвижных единиц с рельсов по причине излома боковой рамы. В ходе предварительного анализа данных были вычислены описательные статистики, была построена матрица корреляции признаков, также был проведен анализ пропусков в данных и построены парные графики. При визуальном анализе парных графиков, а также при анализе описательных статистик было установленно, что выбросов в данных нет. При рассмотрении матрицы корреляции был сделан вывод, что данные в наборе имеют свойство избыточности. Так, например, признаки вес и загрузка имеют сильную корреляцию. Построив оценку функции вероятности количества сошедших вагонов были сделаны предположения, что она имеет пуассоновское распределение, либо геометрическое распределение.

Поскольку с целевым признаком остальные признаки слабо коррелируют, были предложены новые признаки, построенные на основе имеющихся. Удалось построить признак $f_2$, имеющий сильную по сравнению с остальными признаками корреляцию с целевым признаком, при этом слабую корреляцию с остальными признаками. Другие сконструированные признаки $f_1$ и $f_3$ имеют меньшую корреляцию с целевым признаком и значимую корреляцию между собой.

В данной работе была написана программная реализация метода максимального правдоподобия в общем виде. В конструктор данного класса передаются: ссылка на логарифмическую функцию правдоподобия, ссылка на метод оптимизации, граничные условия, ссылка на функцию предсказания, список параметрических функций, список признаковых пространств, название целевой переменной, набор данных.

После были построены регрессионные модели. Для пуассоновской регрессии был выбран набор параметрических функций, обычно в качестве параметрических функций исследователи берут экспоненциальный вид функции, однако были рассмотрены и другие варианты. Аналогично был создан набор параметрических функций для геометрической регрессии. Для обеих регрессий были сформированы признаковые пространства, включающие сконструированные признаки.

Мерой качества был выбран скорректированный критерий Акаике $AIC_c$. Проведя численные эксперименты, были получены следующие результаты:
\begin{description}[font=$\bullet$]
    \item диапазон значений $AIC_c$ для Пуассоновской регрессии:  $[356.87, 567.74]$;
    \item диапазон значений $AIC_c$ для геометрической регрессии: $[153.65, 918.43]$.
\end{description}
Геометрические модели регрессии для заданного набора данных оказалась существенно лучше моделей пуассоновской регрессии. Большинство моделей Пуассоновской регрессии оказались хуже моделей геометрической регрессии по критерию Акаике.